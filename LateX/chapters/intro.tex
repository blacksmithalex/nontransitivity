\chapter{Введенеие}

Фондовый рынок - это неотёемлемая часть экономического сектора уже многие столетия. Область активно развивается и вовлекает все больше и больше людей с каждым годом. Безусловно, для анализа и прогнозирования поведения активов используют методы математического моделирования. В этой работе будет разобрано одно из свойств стратегий, которые могут быть применены для торговли активами на фондовом рынке. Перейдем к проблематике.

\section{Проблематика}

В статье Токарева Сергея Степановича “Нетранзитивный лохотрон на фондовом рынке” утверждается существование нетранзитивных наборов финансовых стратегий$^{[1]}$. Одна стратегия считается лучше другой, если она чаще дает больший результат. Автор утверждает, что существует пример стратегий, при котором первая стратегия лучше второй (в смысле описанном выше), вторая стратегия лучше третьей, но неверно, что первая стратегия лучше стратегии три, то есть:

\begin{center}
($s_1 \succ s_2) \bigwedge (s_2 \succ s_3) \nRightarrow s_1 \succ s_3$
\end{center}

\section{Описание стратегий}

Будем рассматривать 3 стратегии:
\begin{enumerate}
\item стратегия $s_1$(первая): продажа акций производится в самом начале торгового дня по текущей биржевой цене (чистая прибыль всегда равна нулю).
\item стратегия $s_2$ (вторая): продажа акций производится в тот момент, когда их цена впервые снизится более чем на $a\%$, либо поднимется более чем на $b\%$ от их стоимости на начало торгового дня. Если же цена за весь день ни разу не достигнет ни одного из указанных уровней, то продажа производится в конце торгов по текущей цене.
\item стратегия $s_3$ (третья): продажа акций производится в тот момент, когда их цена впервые снизится более чем на $c\%$, либо поднимется более чем на $d\%$ от их стоимости на начало торгового дня. Если опять же цена ни разу за весь день не достигнет ни одного из указанных критических уровней, то продажа производится в конце торгов по текущей цене.
\end{enumerate}

\section{Постановка задачи}
Необходимо дать более точное математическое обоснование данного вопроса, изучить зависимость вероятностей от пороговых значений, разобраться, когда нетранзитивность в этой системе должна проявляться в большей степени, а когда в меньшей. Помимо этого важной задачей будет произвести численный эксперимент на реальных данных и проверить, выполняться ли теоритические выводы на практике.

